\documentclass[10pt]{beamer} %
\usetheme{Boadilla}
\usecolortheme{beaver}
\usepackage[spanish]{babel}
\usepackage[utf8x]{inputenc}
\usefonttheme{professionalfonts}
\usepackage{times}
\usepackage{tikz}
\usepackage{amsmath}
\usepackage{tabulary}
\usepackage{pgfgantt}
\usepackage{verbatim}
\usetikzlibrary{arrows,shapes}
\usepackage{adjustbox}
\usepackage{soul}
\usepackage{listings}
\usepackage{adjustbox}
\usepackage{amsthm}

\newtheorem{Teo}{Teorema}
\newtheorem{prop}{{\it Proposition}}[section]
\setbeamertemplate{itemize item}{\color{red}$\blacksquare$}
\usepackage{hyperref}
\title{avance1}

\title{Hierarchical Bayesian Model for Insurance Claims}
%\institute[Yachay Tech]{School of Mathematical and Computational Sciences, Yachay Tech University, Ecuador}
\author[]{Erika Rivadeneira\\ School of Mathematical and Computational Sciences, Yachay Tech University, Ecuador
\vspace{0.25cm}
\\{\small{\textbf{Tutor:} Saba Rafael Infante\\School of Mathematical and Computational Sciences, Yachay Tech University, Ecuador\\\textbf{Cotutor:} Aracelis Hernández\\Department of Mathematics, Faculty of Science and Technology, University of Carabobo, Venezuela}}}




\date{\today}

\begin{document}
\tikzstyle{every picture}+=[remember picture]
\lstset{}   
\everymath{\displaystyle}

\begin{frame}
	\titlepage
\end{frame}

\begin{frame}
\frametitle{Summary}
\tableofcontents
\end{frame}

%------------------------------------------------------
\section{Introduction}
%------------------------------------------------------

\begin{frame}
\frametitle[9pt]{Introduction}
\begin{itemize}
    \item Insurance companies estimate the risk models to predict the magnitude of the claims and thus be able to determine the premiums that must be paid to the insured
    \item Pricing actuaries must use past information to develop probabilistic models that allow them to model the most important uncertainties involved in the process of losses. 
    \item Insurers must establish a statistical control model that allows for the reduction of unnecessary expenses
    \item Bayesian statistics allows us to include the experience of the experts through the preliminary information that is reflected in the choice of the a priori distributions that each of the parameters will have.
    
\end{itemize}

\end{frame}
%------------------------------------------------------------------------------------
\begin{frame}
\begin{itemize}
    \item In this sense, the purpose is to develop a methodology, under the Bayesian paradigm, allowing predictions to be made of the total future amounts of claims in order to determine the rate of premiums using a Bayesian hierarchical structure.
    \item According to Migon and Moura (2005), the total of claims is related to the insured population and the number of claims of that population during a given period of time.
    \item The people insured in different age groups have different patterns in the frequency and severity of the reported claims
    \item It is introduced an additional category with the idea of being able to describe the regions of residence of the insured. 
    \item This spatial factor comes to represent the combined random effect of many elements on the severity and frequency of claims. 
\end{itemize}
\end{frame}

%---------------------------------------------------------------------------------------------
\section{Theoretical Aspects of the Project}
%-------------------------------------------------------------------------------------------
\begin{frame}
\frametitle[9pt]{Theoretical Aspects of the Project}

\begin{itemize}
    \item \textbf{The Collective Compound Risk Model}\\
    Let $N$ be the total number of claims that arise from a risk in a given period of time and $Z_j$ denote the amount of the $j-$th claim. The total amount of claims is given by
\begin{align}
X=\sum_{j=1}^{N}Z_j,\label{comprisk}
\end{align}
with $X=0$ when $N=0$. The main assumptions of this model are
\begin{itemize}
	\item  The amounts of individual claims $Z_j$ are identically distributed and random independent positive variables.
	\item The total number of claims $N$ is a random variable independent from the amounts of claims $Z_j$.
\end{itemize}
\end{itemize}
\end{frame}
%--------------------------------------------------------------
\begin{frame}
\begin{itemize}
 \item The measure of the expected value (and variance) of the amount of the added claim can be decomposed by measuring the average and the variance of the frequency and severity of the claim, that is,
\begin{align}
&E(X)=E(N)E(Z),\\
&Var(X)=E(N)V(Z)+[E(Z)]^2V(N).\label{var1}
\end{align}
\item \noindent When $N$ follows a Poisson distribution with parameter $\lambda$, it is said that in (\ref{comprisk}) $X$ follows a Poisson distribution composed with parameters $\lambda$ and $F$, where $F(x)=P(Z_1\leq z)$ denotes the function for distributing individual amounts of claims. From (\ref{var1}) it follows that in this case,
\begin{align*}
E(X)&=\lambda E(Z),\\
Var(X)&=\lambda E(Z^2)
\end{align*}

\end{itemize}


    
\end{frame}
%--------------------------------------------------------------
\begin{frame}
    \begin{itemize}
        \item \textbf{The Compound Poisson Process}\\
        Let $X_1,X_2,...,X_n$ be random i.i.d. variables and let $N$ be a random variable independent of the $X_n$ and whose possible values are all integers. Then, the variable $$S_N=\sum_{k=1}^{N}X_k$$ is a compound random variable.\\
    
    \begin{prop}[1]
    \begin{equation}
\label{eq1}
E(g(x))=E(E(g(x)|Y))
\end{equation} 
\end{prop}
 \item From this proposition it can be deduced that
\begin{align*}
E(X)=E(E(X|Y))=\left \{ \begin{matrix} \sum_{k=1}^{\infty}E(X|Y=y_k)P_Y(y_k)
\\\int_{k=1}^{\infty}E(X|Y=y_k)f_Y(y)dy \end{matrix}\right. 
\end{align*}
\end{itemize}
    
\end{frame}
%--------------------------------------------------------------
\begin{frame}
\begin{itemize}
\item and also, $$Var(X)=E(E(X^2|Y))+[E(E(X|Y))]^2.$$ 
\item Now, if $X_1,X_2,…$ are random i.i.d. variables, then it follows that
\begin{align*} 
E(X_k)&=E(X) \quad \forall k=1,2,...\\
Var(X_k)&=Var(X)
\end{align*}
and let $N$ be a random variable independent of the $X_k$ with values ${1,2,…}$, then by (\ref{eq1}) it follows the following proposition
\begin{prop}[2]
\begin{align}
E(S_N)&=E\left(\sum_{k=1}^{N}X_k\right)=E(N)E(X_1)\\
Var(S_N) &= Var\left(\sum_{k=1}^{N}X_k\right)=E(N)Var(X_1)+Var(N)(E(X))^2 \label{var}
\end{align}
\end{prop}\end{itemize}

\end{frame}
%--------------------------------------------------------------
\begin{frame}
\begin{itemize}
\item  Now, consider that $\left\{N(t),t\geq 0\right\}$ is a Poisson process with a rate of $\lambda$ and let $X_1,X_2,...$ be a random i.i.d. variable independent of $\left\{N(t),t\geq 0\right\}$, then the stochastic process $\left\{Y(t),t\geq 0\right\}$ defined as $$Y(t)=\sum_{k=1}^{N(t)}X_k\quad \forall t\geq 0$$ and $$Y(t)=0 \quad si \quad N(t)=0$$ is called the compound Poisson process.\\
\item Assumed that the $X_k$ are i.i.d. and so the two-dimensional stochastic process $${(N(t),Y(t)),t\geq 0}$$ can be considered to retain all the information of interest.\\
Now, using \textbf{proposition 2}:
\begin{align*}
E(Y(t))&=E(N(t))E(X_1)=\lambda tE(X_1)\\
Var(Y(t))&=\lambda tE(X_1^2))
\end{align*}
\end{itemize}
\end{frame}
%--------------------------------------------------------------
\begin{frame}
\begin{prop}[3]
 The stochastic process $\left\{N_i(t),t\geq0\right\}$ is a Poisson process with rates $\lambda_i=\lambda p$; for $i=1,2,...,j$ so then $\left\{N_i(t),t\geq0\right\}$ is an independent Poisson process with rates $\lambda p_{X_1}(i)$ for $i=1,2,…,j$.
\end{prop}
\vspace{1 cm}
\begin{prop}[4]
For $t$ large enough, we can write $$Y(t)\approx N(\lambda t E(X_1),\lambda t E(X^2_1))$$
\end{prop}

\end{frame}
%--------------------------------------------------------------
\begin{frame}
\begin{itemize}
    \item Finally, let $\left\{Y_1(t), t\geq 0\right\}$ and $\left\{Y_2(t), t\geq 0\right\}$ be two compound Poisson processes, defined by $$Y(t)=\sum_{k=1}^{N_i(t)}X_{i,k}\quad \quad \forall t\geq 0  \quad with \quad Y_i(t)=0\quad if\quad  N_i(t)=0$$ where $\left\{N_i(t),t\geq 0\right\}$ is a Poisson process with rate $\lambda_i, i=1,2.$.
    \item Then, the process $\left\{Y(t),t\geq 0\right\}$ defined by $$Y(t)=Y_1(t)+Y_2(t)\quad \quad t\geq 0$$ is also a Poisson process.
\end{itemize}
    
\end{frame}
%--------------------------------------------------------------
\begin{frame}
\begin{itemize}

    \item \textbf{Generalized Exponential Growth Model
    (GEGM)}\\
    \vspace{0.5cm}
    Assume that $\pi_t$, the size of the population in a time interval $t$, characterized by the parameterization $(a,b,\gamma,\lambda)$, is modeled by a probability distribution in the exponential family with average $\mu_t$, that is $$\pi_t\sim Exp(\mu_t)$$ with $$\mu_t=[a+be^{\gamma t}]^{1/\lambda},\quad\quad t\geq0$$
\end{itemize}
\end{frame}
%--------------------------------------------------------------
\section{Presentation of the Problem}
%--------------------------------------------------------------
\begin{frame}{Presentation of the Problem}
\begin{itemize}
    \item Estimate the total number of claims in a specific category under a hierarchical Bayesian structure in a given unit of time that takes into account the insured population, classified according to age, adding a spatial factor that represents the region of residence. 
    \item Under the Bayesian paradigm, the Markov Chain Monte Carlo (MCMC) methods will be used to estimate the parameters of the model from the distributions posterior to each one of them, taking into account the a priori distributions.
    \item Finally, calculate the value of the premiums using different premium principles so that the value of the premium assigned to each insured can be obtained by diving the total value of the premiums by the insured population.
\end{itemize}
\end{frame}
%--------------------------------------------------------------
\section{Objectives}
%--------------------------------------------------------------
\begin{frame}{Objectives}
\begin{itemize}
    \item \textbf{General Objective}\\
    To predict the total number of claims using a hierarchical Bayesian model as a risk measure for insurance companies and to calculate the value of premiums under different principles.\\
    \item\textbf{Specific Objectives}
    \begin{itemize}
        \item Categorize the insured population by age classes in a specific unit of time. Add a spatial factor corresponding to the region of residence that represents the combined random effects of elements that influence the characterization of the claims.
	\item Implement a Markov Chain Monte Carlo (MCMC) algorithm that allows for the estimation of each of the parameters of the proposed model using a priori knowledge.
	\item Based on the number of estimated claims, calculate the value of the premiums according to the different premium principles. 
    \end{itemize}
\end{itemize}
\end{frame}
%--------------------------------------------------------------
\section{Methodology}
%--------------------------------------------------------------
\begin{frame}{Methodology}
\begin{itemize}
\item \textbf{Hierarchical Collective Risk Model}\\
\begin{itemize}
    \item The collective compound risk model is given by $$X_{t,i,a}=\sum_{j=1}^{N_{t,i,a}}Z_{t,i,a,j}$$ with $i=1,..,,I$, $t=1,...,T$ and $a=1,...,A$, in the time interval $(t-1,t)$.
    \item Where $Z_{t,i,a,j}$ is the amount of the $j-$th claim that occurred within the time interval $(t,t-1)$ for an age class $a$ in a region $i$.\\
        \begin{equation}\label{gamma1}
        Z_{t,i,a}\sim Gamma(\kappa_a,\theta_a)\quad with\quad \kappa_a>0,\theta_a>0
    \end{equation}
    and 
    \begin{equation}
    \label{poisson1}
    N_{t,i,a}\sim Poiss(M_{t,i,a}\lambda_a)\quad \lambda_a>0
    \end{equation} 
\end{itemize}
       
where 
\begin{itemize}
    \item $Z_{t,i,a}$ is the individual amount of the claim
    \item $M_{t,i,a}$ is the number of people insured in a time $t$ for an age class $a$ in the region $i$
    \item $\lambda_a$ is the average number of claims per individual per unit of time.
    \item $M_{t,i,a}$ implies a constant population in the time interval $(t-1,t)$ since the growth of the population is not modeled in this model.
\end{itemize}
    \end{itemize}
\end{frame}
%--------------------------------------------------------------
\begin{frame}
\begin{itemize}
    \item Since the sum of these Gamma is also a Gamma, we have that 
    $$X_{t,i,a}|\theta_a,\kappa_a,n_{t,i,a}\sim Gamma(n_{t,i,a}\kappa_a,\theta_a)\quad with\quad \theta_a>0,\kappa_a>a$$
    \item Let's note that in (\ref{gamma1}) we see that the individual claim $Z_{t,i,a}\sim Gamm(\kappa_a,\theta_a)$ is the only variable that depends on age in this model.\\
    \item On the other hand, given that for a certain age class $a$ and region $i$ the insured population $M_ {t, i, a}$ varies over time, then the total number of claims $$\left\{X_{t, i, a},\quad t = 1, ..., T,\quad i = 1, ..., I,\quad a = 1, ..., A\right\}$$ are not identically distributed.
\end{itemize}
    
\end{frame}
\begin{frame}
\begin{itemize}
    \item \textbf{Spatial Effect}\\
    it is assumed that the insured population follows a normal distribution $$M_{t,i,a}\sim N(\mu_{t,i,a},\tau^{-1})\quad con \quad \tau>0$$ with precision $\tau$ and average 
\begin{equation}
	\label{mu_t,i,a}
	\mu_{t,i,a}=\beta_{a_0}+L_i+\beta_{1}e^{t\beta_{a_2}}
\end{equation}
 where $L_i$ represents the spatial factor for the region $i$ and $\beta_{a_0},\beta_{a_2}$ are parameters related to age. It's important to mention that the age of the insured population is independent of the information related to the region. 
\end{itemize}
    
\end{frame}
%--------------------------------------------------------------
\begin{frame}{Priori Distributions}
\begin{align*}
\lambda_{a}&\sim Gamm(\alpha_{\lambda},\beta_{\lambda})\\
\theta_{a}&\sim Gamm(\alpha_{\theta},\beta_{\theta})\\
\kappa_{a}&\sim Gamm(\alpha_{	\kappa},\beta_{	\kappa})
\end{align*}
These distributions are independent with $\alpha_{\lambda},\beta_{\lambda},\alpha_{\theta},\beta_{\theta},\alpha_{	\kappa},\beta_{	\kappa}$ non-negative amounts.
\begin{enumerate}[a)]
	\item Distributions that describe the value of the claims, the number of claims and the insured population:
	\begin{align*}
	X_{t,i,a}\sim Gamm(n_{t,i,a}\kappa_{a},\theta_a),\quad \theta_a>0,\kappa_a>0,\\
	N_{t,i,a}\sim Poiss(M_{t,i,a}\lambda_a),\quad \lambda_a>0,\\
	M_{t,i,a}\sim N(\mu_{t,i,a},\tau^{-1}),
	\end{align*}
	where $$\mu_{t,i,a}=\beta_{a_0}+L_i+\beta_1e^{t\beta_{a_2}}$$
	
\end{enumerate}

\end{frame}
%--------------------------------------------------------------
\begin{frame}{Priori Distributions}
\begin{enumerate}[b)]
   \item Distributions that describe the age and the regions
	\begin{align*}
	\theta_{a}&\sim Gamm(\alpha_{\theta},\beta_{\theta}),\\
	\kappa_{a}&\sim Gamm(\alpha_{	\kappa},\beta_{	\kappa}),\\
	\lambda_{a}&\sim Gamm(\alpha_{\lambda},\beta_{\lambda}),\\
	\beta_{a_0}=\beta_0+\varepsilon_a^0,&\textrm{ with }\varepsilon_a^0\sim N(0,\tau_{\varepsilon_0}^{-1}),\\
	\beta_{a_2}=\beta_2+\varepsilon_a^2,& \textrm{ with }\varepsilon_a^2\sim N(0,\tau_{\varepsilon_2}^{-1}),\\
	L&\sim NMV(0,\tau^{-1}Q^{-1}),
	\end{align*}
	where 
	\begin{align*}
	Q_{gh}=\left\{\begin{matrix}
	1+|\eta|\cdot m_g, \quad &g=h\\
	-\eta, \quad &g\neq h,g,h=1,2,...,I\\
	0,\quad &\textrm{otherwise}
	\end{matrix}\right.
	\end{align*}
	    
\end{enumerate}
    
\end{frame}
%--------------------------------------------------------------
\begin{frame}{Priori Distributions}
\begin{enumerate}[c)]
\item Distributions of the a priori hyperparameters
	\begin{align*}
	\beta_0\sim N(\mu_0,\tau_0^{-1})\\
	\beta_1\sim N(\mu_1,\tau_1^{-1})\\
	\beta_2\sim N(\mu_2,\tau_2^{-1})\\
	f(\eta)=\frac{1}{(1+\eta)^2},\quad \eta>0
	\end{align*}
	and $\psi=(\tau,\sigma,\tau_{\varepsilon_0},\tau_{\varepsilon_2},\alpha_{\theta},\beta_{\theta},\alpha_{\lambda},\beta_{\lambda},\alpha_{	\kappa},\beta_{	\kappa})$ follow Gamma distributions with known parameters $$\Rightarrow \psi \sim Gamm(\alpha_\psi,\beta_\psi)$$ where $\mu_0,\mu_1,\mu_2,\tau_0,\tau_1,\tau_2,\alpha_\psi,\beta_\psi$ are known values.
 \end{enumerate}   
\end{frame}
%--------------------------------------------------------------
\begin{frame}{Posterior Distributions}
$$P(\theta_{a}|\Theta_{-\theta_{a}},D_{t})\propto Gamm(\alpha_{\theta}+\sum_{t=1}^{T}\sum_{i=1}^{I}n_{t,i,a},\sum_{t=1}^{T}\sum_{i=1}^{I}X_{t,i,a}+\beta_{\theta})\quad\textrm{for }a=1,...,A$$
$$P(\lambda_{a}|\Theta_{-\lambda_{a}},D_{t})\propto Gamm(\alpha_{\lambda}+\sum_{t=1}^{T}\sum_{i=1}^{I}n_{t,i,a},\sum_{t=1}^{T}\sum_{i=1}^{I}M_{t,i,a}+\beta_{\lambda})\quad\textrm{for }a=1,...,A$$
\begin{align*}P(\beta_{0}|\Theta_{-\beta_{0}},D_{t})\propto& N\left(\frac{\tau_0\mu_0+\tau\sum_{t=1}^{T}\sum_{i=1}^{I}\sum_{a=1}^{A}\left( M_{t,i,a}-\varepsilon_a^0-L_i-\beta_{1}e^{t\beta_{a_2}}\right)}{\tau TIA+\tau_0},\\ 
&(\tau TIA+\tau_0)^{-1} \right)\\
P(\beta_{1}|\Theta_{-\beta_{1}},D_{t})\propto& N\left(\frac{\tau_1\mu_1+\tau\sum_{t=1}^{T}\sum_{i=1}^{I}\sum_{a=1}^{A}(M_{t,i,a}-\beta_{0}-\varepsilon_a^0-L_i)e^{t(\beta_{2}+\varepsilon_a^2)}}{\tau_1+\tau\sum_{t=1}^{T}\sum_{i=1}^{I}\sum_{a=1}^{A}e^{2t\beta_{a_2}}}
\\&,\left(\tau_1+\tau\sum_{t=1}^{T}\sum_{i=1}^{I}\sum_{a=1}^{A}e^{2t(\beta_{2}+\varepsilon_a^2)} \right)^{-1}\right)    
\end{align*}

\end{frame}
%--------------------------------------------------------------
\begin{frame}{Posterior Distributions}
\begin{align*}
    &P(\varepsilon_a^0|\Theta_{-\varepsilon_a^0},D_{t})\propto N\left( \frac{\tau\sum_{t=1}^{T}\sum_{i=1}^{I}\left(M_{t,i,a}-\beta_{0}-L_i-\beta_{1}e^{t(\beta_{2}+\varepsilon_a^2)}\right)}{(\tau TI+\tau_{\varepsilon_0})},(\tau TI+\tau_{\varepsilon_0})^{-1} \right)\\
    &P(\beta_2|\Theta_{-\beta_{2}},D_t)\propto e^{-\frac{\tau_2}{2}(\beta_{2}^2-2\beta_{2}\mu_2)-\frac{\tau}{2}\sum_{t=1}^{T}\sum_{i=1}^{I}\sum_{a=1}^{A}\left(\beta_{1}e^{t(\beta_2+\varepsilon_a^2)}-(M_{t,i,a}-\beta_{a_0}-L_i)\right)^2}\\
    &P(\varepsilon_a^2|\Theta_{-\varepsilon_a^2},D_{t})\propto e^{-\frac{\tau_{\varepsilon_2}}{2}(\varepsilon_a^2)^2-\frac{\tau}{2}\sum_{t=1}^{T}\sum_{i=1}^{I}\left(\beta_{1}e^{t(\beta_{2}+\varepsilon_a^2)}-(M_{t,i,a}-\beta_{a_0}-L_i) \right)^2}\\ &\textrm{for }a=1,...,A\\
    &P(\tau|\Theta_{-\tau},D_{t})\propto Gamm\left(\frac{1}{2}TIA+\alpha_\tau, \frac{1}{2}\sum_{t=1}^{T}\sum_{i=1}^{I}\sum_{a=1}^{A}(M_{t,i,a}-\mu_{t,i,a})^2+\beta_\tau\right)\\
    &P(\tau_{\varepsilon_0}|\Theta_{-\tau_{\varepsilon_0}},D_{t})\propto Gamm\left(\frac{A}{2}+\alpha_{\tau_{\varepsilon_0}},\beta_{\tau_{\varepsilon_0}}+\frac{1}{2}\sum_{a=1}^{A}(\varepsilon_a^0)^2 \right)
\end{align*}
    
\end{frame}
%--------------------------------------------------------------
\begin{frame}{Posterior Distributions}
    \begin{align*}
        &P(\tau_{\varepsilon_2}u|\Theta_{-\tau_{\varepsilon_2}},D_{t})\propto Gamm\left(\frac{A}{2}+\alpha_{\tau_{\varepsilon_2}},\beta_{\tau_{\varepsilon_2}}+\frac{1}{2}\sum_{a=1}^{A}(\varepsilon_a^2)^2 \right)\\
        &P(\alpha_\theta|\Theta_{-\alpha_\theta},D_{t})\propto \Gamma(\alpha_{\theta})^{-A}\beta_{\theta}^{A\alpha_{\theta}}\alpha_{\theta}^{\alpha_{\alpha_{\theta}}-1}e^{-\beta_{\alpha_{\theta}}\alpha_{\theta}}\left(\prod_{a=1}^{A}\theta_{a}^{\alpha_{\theta}-1}\right)\\
        &P(\alpha_\lambda|\Theta_{-\alpha_\lambda},D_{t})\propto \Gamma(\alpha_{\lambda})^{-A}\beta_{\lambda}^{A\alpha_{\lambda}}\alpha_{\lambda}^{\alpha_{\alpha_{\lambda}}-1}e^{-\beta_{\alpha_{\lambda}}\alpha_{\lambda}}\left(\prod_{a=1}^{A}\lambda_{a}^{\alpha_{\lambda}-1}\right)\\
        &P(\beta_{\theta}|\Theta_{-\beta_{\theta}},D_{t})\propto Gamma\left(A\alpha_{\theta}+\alpha_{\beta_{\theta}},\beta_{\beta_{\theta}}+\sum_{a=1}^{A}\theta_{a} \right)\\
        &P(\beta_{\lambda}|\Theta_{-\beta_{\lambda}},D_{t})\propto Gamma\left(A\alpha_{\lambda}+\alpha_{\beta_{\lambda}},\beta_{\beta_{\lambda}}+\sum_{a=1}^{A}\lambda_{a} \right)\\
    \end{align*}
\end{frame}
%--------------------------------------------------------------
\begin{frame}{Posterior Distributions}
For the spatial variable, we have to study $ mg $ neighboring regions, then the variance-covariance matrix of the multivariate normal distribution is given by
\begin{align*}
	\sigma^{-1}\cdot\left( \begin{array}{cc}
	1+|\eta|\cdot mg & -\eta \\
	-\eta & 1+|\eta|\cdot mg
	\end{array} \right)^{-1} =\left( \begin{array}{cc}
	P&S\\
	S&P
	\end{array}\right)
\end{align*}
whence, the correlation coefficient is given by 
$$\rho =\frac{\eta}{1+\eta}$$
So that,
\begin{align*}
&P(L_i|\Theta_{-L_i},D_{t})\propto\left[\prod_{t=1}^{T}\prod_{a=1}^{A}e^{-\frac{\tau}{2}(M_{t,i,a}-\mu_{t,i,a})^2}\right]\cdot e^{-\frac{1}{2P^2(1-\rho^2)\left(L_1^2+L_2^2-2\rho L_1L_2\right)}}\\
    &P(\eta|\Theta_{\eta},D_{t})\propto \frac{1}{(1+\eta)^2P\sqrt{1-\rho^2}}e^{-\frac{1}{2P^2(1-\rho^2)}(L_1^2+L_2^2-2\rho L_1L_2)}\\
    &P(\sigma|\Theta_{-\sigma},D_{t})\propto \frac{1}{P^2}\sigma^{\alpha_\sigma-1}e^{-\sigma \beta_\sigma}e^{-\frac{1}{2P^2(1-\rho^2)}(L_1^2+L_2^2-2\rho L_1L_2)}
\end{align*}

\end{frame}
%--------------------------------------------------------------
\section{Current state of the research project}
%--------------------------------------------------------------
\begin{frame}{Simulation Studies and Model Fitting}
\begin{itemize}
    \item Simulation of the history of insurance claims data
    \item Selection of parameters' values. If the parameter values are properly selected, it could hopefully contain important practical meaning to understand easily the elements of the proposed model.
\item  Data on the number, total amount of insurance and the region and age class of the insured:
    \begin{itemize}
	\item \textbf{Claim Frecuency Parameters}
	\begin{itemize}
		\item $\lambda_{a}\sim Gamma(40,200)$ for any $a=1,2,...,7$
	\end{itemize}
	\item \textbf{Claim Severity Parameters}
	\begin{itemize}
		\item $\kappa_{a}=1$ for any $a=1,2,...,7$
		\item $\theta_{a}\sim Gamma(400,10000)$ for any $a=1,2,...,7$ 
	\end{itemize}
	\item \textbf{Population Parameters}
	\begin{itemize}
		\item $\beta_{0}=50$
		\item $\beta_{1}=20$
		\item $\beta_{2}=0.075$
		\item $\eta\sim Pareto(1,1)$	
	\end{itemize}
\end{itemize}
\end{itemize}


\end{frame}
%--------------------------------------------------------------
\section{References}
%--------------------------------------------------------------
\begin{frame}{References}
    \begin{itemize}
    \item Dimakos, X. and Frigessi, A. (2002). Bayesian premium rating with latent structure. Scan-
dinavian Actuarial Journal, 2002(3):162--184. 
        \item Gamerman, D. (1997). Markov Chain Monte Carlo: Stochastic Simulation for Bayesian
Inference. CRC Press.
\item Migon, H. and Gamerman, D. (1993). Generalized exponential growth models-a Bayesian approach. Journal of Forecasting, 12(7):573--584.
\item Migon, H. and Moura, F. (2005). Hierarchical Bayesian collective risk model: an application to health insurance. Insurance: Mathematics and Economics, 36(2):119--135.
\item Migon, H. and Penna, E. (2006). Bayesian analysis of a health insurance model. Journal of Actuarial Practice, 13:61--80.
\item Pettitt, A. N., Weir, I. S., and Hart, A. G. (2002). A conditional autoregressive gaussian
process for irregularly spaced multivariate data with application to modelling large
sets of binary data. Statistics and Computing, 12(4):353--367.
\item Souza, D., Moura, F., and Migon, H. (2009). Small area population prediction via hierar-
chical models. In Survey Methodology, volume 39 of Catalogue No. 12-001-X, pages
203--214. Statistics Canada.
    \end{itemize}
\end{frame}

\end{document}
